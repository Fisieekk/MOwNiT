\documentclass{article}
\usepackage{graphicx} % Required for inserting images
\usepackage{listings}
\usepackage{caption}
\usepackage{color}
\usepackage{adjustbox}
\usepackage[T1]{fontenc}
\usepackage{enumitem}
\usepackage{amsmath}
\captionsetup[table]{name=\textbf{Tabela}}
\setcounter{MaxMatrixCols}{14}
\usepackage[a4paper, total={6in, 8in}]{geometry}
\usepackage[dvipsnames]{xcolor}
\definecolor{codegreen}{rgb}{0,0.6,0}
\definecolor{codegray}{rgb}{0.5,0.5,0.5}
\definecolor{codepurple}{rgb}{0.58,0,0.82}
\definecolor{backcolour}{rgb}{0.95,0.95,0.92}
\usepackage{hyperref}
\hypersetup{
    colorlinks=true,
    linkcolor=blue,
    filecolor=magenta,      
    urlcolor=cyan,
    pdftitle={Overleaf Example},
    pdfpagemode=FullScreen,
    }
\renewcommand{\figurename}{\textbf{Wizualizacja}}
\usepackage{booktabs}
\lstset{frame=tb,
  language=Python,
  backgroundcolor=\color{backcolour},   
  commentstyle=\color{codepurple},
  keywordstyle=\color{NavyBlue},
  numberstyle=\tiny\color{codegray},
  stringstyle=\color{codepurple},
  basicstyle=\ttfamily\footnotesize\bfseries,
  breakatwhitespace=false,         
  breaklines=true,                 
  captionpos=t,                    
  keepspaces=true,                 
  numbers=left,                    
  numbersep=5pt,                  
  showspaces=false,                
  showstringspaces=false,
  showtabs=false,                  
  tabsize=2
}

\title{
    Marek Małek, Marcin Serafin 14.03.2024 \\ 
    \Large Laboratorium 02 \\
    Metoda najmniejszych kwadratów
    }
\date{\vspace{-5ex}}
\begin{document}


\maketitle

\section{Treść zadania}
Celem zadania było zastosowanie metody najmniejszych kwadratów do predykcji, czy nowotwór jest złośliwy czy łagodny. Do rozwiązania należało wykorzystać bibliotekę \textbf{pandas} oraz typ \textbf{DataFrame}. Dostarczone zostały dwa zbiory danych:
\begin{itemize}
    \item \textbf{breast-cancer-train.dat}
    \item \textbf{breast-cancer-validate.dat}
\end{itemize}
Oraz plik \textbf{breast-cancer.labels}, w którym zostały zawarte nazwy kolumn.

\section{Rozwiązanie zadania}

\subsection{Wczytanie danych}

W celu wczytania danych wykonano następujący fragment kodu:
\begin{lstlisting}
column_names = []

with open("data\\breast-cancer.labels") as f:
    for line in f.readlines():
        column_names.append(line[:(len(line))-1])

breast_cancer_train = pd.io.parsers.read_csv("data\\breast-cancer-train.dat")
breast_cancer_train.columns = column_names

breast_cancer_validate =  pd.io.parsers.read_csv("data\\breast-cancer-validate.dat")
breast_cancer_validate.columns = column_names

\end{lstlisting}

Dane zostały wczytane do odpowiednich zmiennych oraz przypisane zostały nazwy charakterystyk do kolumn \textbf{DataFramea}

\pagebreak

Widok pierwszych pięciu wierszy i pierwszych pięciu kolumn \textbf{DataFramea} zbioru \\ danych \textbf{breast\_cancer\_train}:



\begin{tabular}{lrlrrr}
\toprule
 & patient ID & Malignant/Benign & radius (mean) & texture (mean) & perimeter (mean) \\
\midrule
0 & 842517 & M & 20.570000 & 17.770000 & 132.900000 \\
1 & 84300903 & M & 19.690000 & 21.250000 & 130.000000 \\
2 & 84348301 & M & 11.420000 & 20.380000 & 77.580000 \\
3 & 84358402 & M & 20.290000 & 14.340000 & 135.100000 \\
4 & 843786 & M & 12.450000 & 15.700000 & 82.570000 \\
\bottomrule
\end{tabular}

\subsection{Wizualizacja pojedyńczej charakterystki}

\subsubsection{Histogram}
Stworzono histogram klasyfikujący wśród ilu pacjentów wykryto nowotwór z danym średnim promieniem.

\begin{figure}[!h]
    \centering
    \includegraphics[scale=0.6]{img/hist.png}
    \caption{Historgram charakterystki radius (mean)}
    \label{fig:enter-label}
\end{figure}

\pagebreak

\subsubsection{Wykres}

Analogicznie do poprzedniego podpunktu stworzono wykres tej samej charakterystki.

\begin{figure}[!h]
    \centering
    \includegraphics[scale=0.6]{img/plot.png}
    \caption{Wykres charakterystki radius (mean)}
    \label{fig:enter-label}
\end{figure}

\subsection{Reprezentacje danych}

Stworzono reprezentacje danych zawartych w zbiorach w oparciu o wzory:

Reprezentacja liniowa:
\begin{equation}
A_{\text{lin}} = \begin{bmatrix} 
f_{1,1} & f_{1,2} & \hdots & f_{1,m} \\
f_{2,1} & f_{2,2} & \hdots & f_{2,m} \\
f_{3,1} & f_{3,2} & \hdots & f_{3,m} \\
\vdots  & \vdots  & \ddots &         \\
f_{n,1} & f_{n,2} &        & f_{n,m}
\end{bmatrix}
\end{equation}

Reprezentacja kwadratowa (wybrano 4 parametry: \textbf{radius (mean)}, \textbf{perimeter (mean)}, \\  \textbf{area (mean)}, \textbf{symmetry (mean)}):
\small
\setlength\arraycolsep{1.5pt}
\begin{equation}
A_{\text{quad}} = \\ \begin{bmatrix} 
f_{1,1} & f_{1,2} & f_{1,3} & f_{1,4} & f_{1,1}^2 & f_{1,2}^2 & f_{1,3}^2 & f_{1,4}^2 & f_{1,1}f_{1,2} & f_{1,1}f_{1,3} & f_{1,1}f_{1,4} & f_{1,2}f_{1,3} & f_{1,2}f_{1,4} & f_{1,3}f_{1,4} \\
\vdots & \vdots & \vdots & \vdots & \vdots & \vdots & \vdots & \vdots & \vdots & \vdots & \vdots & \vdots & \vdots & \vdots \\
f_{n,1} & f_{n,2} & f_{n,3} & f_{n,4} & f_{n,1}^2 & f_{n,2}^2 & f_{n,3}^2 & f_{n,4}^2 & f_{n,1}f_{n,2} & f_{n,1}f_{n,3} & f_{n,1}f_{n,4} & f_{n,2}f_{n,3} & f_{n,2}f_{n,4} & f_{n,3}f_{n,4} \\
\end{bmatrix}
\end{equation}

\subsection{Wektor $b$ oraz wagi reprezentacji}

W celu znalezienia wag dla liniowych i kwadratowych reprezentacji użyto równania:

\begin{equation}
    \boldmath
    A^TAw = A^Tb
\end{equation}

gdzie wektor $b$ jest zadany jako:
\begin{equation}
\begin{bmatrix}
    \alpha_1 \\
    \alpha_2 \\
    \vdots \\ 
    \alpha_n
\end{bmatrix}
\text{ gdzie } \alpha_i = \begin{cases}
1 & \text{jeśli nowotwór jest złośliwy} \\
-1 & \text{wpp.}
\end{cases}
\end{equation}

Przy tworzeniu wektora $b$ użyto funkcji \textbf{np.where}

\begin{lstlisting}
    b_train = np.where(breast_cancer_train['Malignant/Benign'] == 'M', 1, -1)
    b_validate = np.where(breast_cancer_validate['Malignant/Benign'] == 'M', 1, -1)
\end{lstlisting}

\subsection{Współczynnik uwarunkowania macierzy}

Przy obliczeniu współczynnika uwarunkowania macierzy $\text{cond}(A)$ oraz $\text{cond}(A^TA)$ wykorzystano równania:

\begin{equation}
    \text{cond}(A) = ||A|| \cdot ||A^T||
\end{equation}

\begin{equation}
    \text{cond}(A^TA) = \text{cond}(A)^2
\end{equation}

\textbf{DataFrame} reprezentujący obliczone wartości:
\begin{table}[!h]
    \begin{adjustbox}{width=420pt,center}
    \centering
    \begin{tabular}{l|r|r}
    \toprule
     & cond($A$) & cond($A^TA$) \\
    \midrule
    reprezentacja liniowa & 3392717729702346240.000000 & 11518506744199914717184.000000 \\
    reprezentacja najmniejszych kwadratów & 256020503885671.218750 &    902853865047191680.000000 \\
    \bottomrule
    \end{tabular}
    \end{adjustbox}
    \caption{DataFrame współczynników uwarunkowania macierzy}
    \label{tab:my_label}
\end{table}

\subsubsection{Przewidywanie nowotworu}

W celu określenia czy dany nowotwór jest złośliwy czy nie pomnożono reprezentacje liniową oraz kwadratową przez uprzednio wyliczone odpowienie wektory wag. Dla otrzymanego wektora $p$ zliczono liczbę wartości $p_i$ takich, że $p[i]\leq0$ (wtedy nowotwór prawdopodobnie był łagodny) oraz $p[i]>0$ (wtedy prawdopodobnie nowotwór był złośliwy). Obliczono też liczbę przewidywań fałszywie dodatnich (przewidziano, że nowotwór był złośliwy, a tak naprawdę był łagodny) oraz fałszywie ujemnych (analogicznie do poprzedniego przykładu). Otrzymane wyniki zestawiono w \textbf{DataFramie}
\begin{table}[!h]
    \begin{adjustbox}{width=420pt,center}
    \centering
    \begin{tabular}{l|r|r|r|r}
    \toprule
     & Liczba fałszywie ujemnych & Liczba fałszywie dodatnich & \shortstack{Liczba przewidzianych \\ nowotworów złośliwych} & \shortstack{Liczba przewidzianych \\ nowotworów łagodnych} \\
    \midrule
    Reprezentacja liniowa & 2 & 6 & 63 & 196 \\
    Reprezentacja najmnieszych kwadratów & 5 & 15 & 69 & 190 \\
    \bottomrule
    \end{tabular}
    \end{adjustbox}
    \caption{DataFrame klasyfikacji fałszywie ujemnych i fałszywie dodatnich}
    \label{tab:my_label}
\end{table}

\section{Wnioski}





\section{Bibliografia}

\begin{enumerate}
    \item \url{http://heath.cs.illinois.edu/scicomp/notes/cs450_chapt03.pdf}
    \item \url{https://pythonnumericalmethods.berkeley.edu/notebooks/chapter16.00-Least-Squares-Regression.html}
\end{enumerate}


\end{document}
